\chapter*{}
\addcontentsline{toc}{chapter}{Abstract}
\begin{center}
  \Large\textbf{Abstract}
\end{center}

\noindent
This thesis investigates the deployment and performance evaluation of \textit{Aether}, an open-source, enterprise-grade private 5G platform developed by the Open Networking Foundation (ONF), within virtualized environments. The study explores the feasibility of deploying Aether in constrained setups such as single-node or dual-node virtual machines, which are common in academic and small enterprise scenarios. Two configurations were examined: a Quick Start deployment using \texttt{gNBsim} for validating core functionality, and a full Aether deployment incorporating multiple blueprints, runtime control via ROC, and Quality of Service testing using \texttt{UERANSIM}. The results demonstrate Aether’s modularity, resilience under load, and suitability for reproducible research and educational use. Additionally, the study identifies deployment challenges, including CPU and memory bottlenecks, multi-interface networking complexity, and telemetry limitations, while also contributing enhancements to monitoring and provisioning mechanisms. Overall, this work advances the understanding of scalable, cost-effective private 5G solutions and serves as a reference model for future academic experimentation and enterprise deployment.
